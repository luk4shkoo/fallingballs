\documentclass[12pt,a4paper]{article}
\usepackage[utf8]{inputenc}
\usepackage[slovak]{babel}
\usepackage{tikz}
\usepackage{graphicx}
\usetikzlibrary{shapes,arrows,positioning}

\title{Dokumentácia k projektu FallingBalls}
\author{Lukáš Belavý}
\date{\today}

\begin{document}
\maketitle

\section{Zámer projektu}
FallingBalls je hra, kde hráč ovláda paddle (pádlo) na spodku obrazovky a snaží sa zostreliť padajúce loptičky pred tým, než sa dotknú pádla. Hra implementuje rôzne typy loptičiek s unikátnym správaním:
\begin{itemize}
    \item Normálne loptičky (červené) - základný typ
    \item Rýchle loptičky (modré) - pohybujú sa rýchlejšie
    \item Deliteľné loptičky (fialové) - pri zásahu sa rozdelia na dve menšie
    \item Odolné loptičky (zelené) - vyžadujú dva zásahy na zničenie
\end{itemize}

Cieľom je dosiahnuť čo najvyššie skóre zostreľovaním loptičiek, pričom každý typ má inú bodovú hodnotu.

\section{Diagram tried hierarchie dedenia}
\begin{center}
\begin{tikzpicture}[
    node distance=2cm,
    class/.style={
        rectangle,
        draw,
        rounded corners=3pt,
        text width=4cm,
        minimum height=1.5cm,
        text centered,
        fill=white
    },
    arrow/.style={->,>=stealth,thick}
]
% Abstract base class
\node[class, fill=gray!10] (ball) {
    \textbf{Ball}\\
    \footnotesize{move(), draw(), onHit()}
};

% Second layer classes
\node[class] (normal) [below left=2cm and 1cm of ball] {
    \textbf{NormalBall}\\
    \footnotesize{Základná implementácia}\\
    \footnotesize{(červená farba)}
};

\node[class] (tough) [below=2cm of ball] {
    \textbf{ToughBall}\\
    \footnotesize{Prekonáva onHit(), draw()}\\
    \footnotesize{(zelená farba, 2 zásahy)}
};

\node[class] (split) [below right=2cm and 1cm of ball] {
    \textbf{SplitBall}\\
    \footnotesize{Prekonáva onHit()}\\
    \footnotesize{(fialová farba, delenie)}
};

% Third layer - FastBall
\node[class] (fast) [below=2cm of normal] {
    \textbf{FastBall}\\
    \footnotesize{Modifikuje rýchlosť}\\
    \footnotesize{(modrá farba)}
};

% Draw inheritance arrows
\draw[arrow] (normal.north) -- (ball.south west);
\draw[arrow] (tough.north) -- (ball.south);
\draw[arrow] (split.north) -- (ball.south east);
\draw[arrow] (fast.north) -- (normal.south);

\end{tikzpicture}
\end{center}

\section{OOP črty v programe}

\subsection{1. Dedenie}
Program využíva viacúrovňovú hierarchiu dedenia:
\begin{itemize}
    \item Abstraktná trieda \texttt{Ball} definuje základné správanie všetkých loptičiek
    \item Konkrétne triedy (\texttt{NormalBall}, \texttt{FastBall}, \texttt{ToughBall}, \texttt{SplitBall}) dedia od \texttt{Ball}
    \item Každá podtrieda rozširuje alebo modifikuje správanie základnej triedy
\end{itemize}

\subsection{2. Zapuzdrenie}
\begin{itemize}
    \item Všetky atribúty tried sú deklarované ako private alebo protected
    \item Prístup k atribútom je kontrolovaný cez gettery a settery
    \item Príklad v triede \texttt{Paddle}: private atribúty \texttt{x}, \texttt{width}, \texttt{height}, \texttt{speed}
\end{itemize}

\subsection{3. Polymorfizmus}
\begin{itemize}
    \item Metóda \texttt{onHit()} je implementovaná rôzne v každej podtriede
    \item \texttt{ToughBall} vyžaduje dva zásahy
    \item \texttt{SplitBall} sa rozdelí na menšie loptičky
    \item \texttt{GameManager} pracuje s loptičkami polymorfne cez referenciu na \texttt{Ball}
\end{itemize}

\subsection{4. Abstrakcia}
\begin{itemize}
    \item Abstraktná trieda \texttt{Ball} definuje rozhranie pre všetky loptičky
    \item Trieda \texttt{GameManager} abstrahuje hernú logiku
    \item \texttt{CollisionManager} abstrahuje detekciu kolízií
\end{itemize}

\section{Implementačné detaily}
\subsection{Hlavné triedy}
\begin{itemize}
    \item \texttt{Ball} - Abstraktná trieda definujúca základné vlastnosti loptičiek
    \item \texttt{GameManager} - Spravuje hernú logiku, kolízie a stav hry
    \item \texttt{GamePanel} - Zobrazuje hru a spracováva užívateľský vstup
    \item \texttt{Paddle} - Reprezentuje hráčom ovládané pádlo
    \item \texttt{Bullet} - Reprezentuje projektily vystrelené z pádla
\end{itemize}

\subsection{Herné mechaniky}
\begin{itemize}
    
    \item Rôzne typy loptičiek s unikátnym správaním
    \item Bodový systém závislý od typu zničenej loptičky
    \item Kolízny systém medzi loptičkami, projektilmi a pádlom
\end{itemize}


\end{document} 